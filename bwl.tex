%coding:utf-8

%----------------------------------------
%FOSAPHY, a LaTeX-Code for a summary of optics
%Copyright (C) 2014, Mario Felder, Michael Fallegger

%This program is free software; you can redistribute it and/or
%modify it under the terms of the GNU General Public License
%as published by the Free Software Foundation; either version 2
%of the License, or (at your option) any later version.

%This program is distributed in the hope that it will be useful,
%but WITHOUT ANY WARRANTY; without even the implied warranty of
%MERCHANTABILITY or FITNESS FOR A PARTICULAR PURPOSE.  See the
%GNU General Public License for more details.
%----------------------------------------
\chapter{Grundlagen}
\[
	\textbf{Produktivität}=\frac{\textbf{Arbeitsergebnis}}{\textbf{Einsatz von Produktionsfaktoren}}
\]
\\
\[
	\textbf{Wirtschaftlichkeit}=\frac{\textbf{Ertrag}}{\textbf{Aufwand}}
\]
\\
\[
	\textbf{Rentabilität}=\frac{\textbf{Gewinn}}{\textbf{$\oslash$ eingesetztes Kapital}}
\]
\\
\chapter{Kostenrechnung}
\section{Liquidität}
\[
		\textbf{Cash-Ratio o. Liquiditätsgrad I}=\frac{\textbf{Liquide Mittel}}{\textbf{kurzf. Fremdkapitel}}
\]
\\
\[
	\textbf{Quick-Ratio o. LG II }=\frac{\textbf{Liquide Mittel + Forderungen}}{\textbf{kurzf. Fremdkapitel}}
\]
\\
\[
		\textbf{Current-Ratio}=\frac{\textbf{Umlaufvermögen}}{\textbf{kurzf. Fremdkapitel}}
\]
\\
\section{Vermögensstruktur}
\[
		\textbf{Umlaufintensität}=\frac{\textbf{Umlaufvermögen}}{\textbf{Gesamtvermögen}}
\]
\\
\[
		\textbf{Anlageintensität}=\frac{\textbf{Anlagevermögen}}{\textbf{Gesamtvermögen}}
\]
\\
\section{Kapitalstruktur}
\[
		\textbf{Fremdfinanzierungsgrad}=\frac{\textbf{Fremdkapital}}{\textbf{Gesamtkapital}}
\]
\\
\[
		\textbf{Eigenfinanzierungsgrad}=\frac{\textbf{Fremdkapital}}{\textbf{Gesamtkapital}}
\]
\\
\section{Deckung der Anlagen}
\[
		\textbf{Anlagendeckungsgrad I}=\frac{\textbf{Eigenkapital}}{\textbf{Anlagevermögen}}
\]
\\
\[
		\textbf{Anlagendeckungsgrad II}=\frac{\textbf{Eigenkapital + langfr. Fremdkapital}}{\textbf{Anlagevermögen}}
\]
\\
\section{Rentabilität}
\[
		\textbf{Eigenkapitalrentabilität}=\frac{\textbf{Gewinn}}{\textbf{durchschn. Eigenkapital}}
\]
\\
\[
		\textbf{Gesamtkapitalrentabilität}=\frac{\textbf{Gewinn + Fremdkapitalzinsen}}{\textbf{durchschn. Eigenkapital}}
\]
\\
\chapter{Kosten- und Investitionsrechnung}
\section{Amortisationsrechnung}
\[
		\textbf{Amortisation}_\textbf{Erweiterung}=\frac{\textbf{Kapitaleinsatz}}{\textbf{Gewinn + Abschreibungen}}
\]
\\
\[
		\textbf{Amortisation}_\textbf{Rationalisierung}=\frac{\textbf{Kapitaleinsatz}}{\textbf{Kosteneinsparung + Abschreibungen}}
\]
\\
\section{Kapitalwertmethode}
\[
	a_n=\sum_{t=1}^{n}\left( \frac{1}{\left( 1+i\right) ^t} \right) 
\]
\[
	K_0=\sum_{t=0}^{n}\left( \frac{g_t}{\left( 1+i\right) ^t} \right) + \frac{L_n}{\left( 1+i\right) ^n}-I_0
\]
Fallen die Einzahlungsüberschüsse gleichmässsig über die gesamte Nutzungsdauer an, so kann eine vereinfachte Berechnung vorgenommen werden.\\
\[
	K_0=a_{\bar{n}}\cdot \frac{L_n}{\left( 1+i\right) ^n}-I_0
\]
\\
\section{Methode des internen Zinssatzes}
\[
	I_0=\sum_{t=0}^{n}\left( \frac{g_t}{\left( 1+i\right) ^t} \right) + \frac{L_n}{\left( 1+i\right) ^n}
\]
\\
\section{Annuitätenmethode}
\[
	A=\frac{1}{a_{\bar{n}}}K_0
\]
\\
\chapter{Materialwirtschaft}
\section{Kontrollmöglichkeiten}
\[
		\textbf{Lieferbereitschaftsgrad}=\frac{\textbf{sofort ausgelieferte Menge}}{\textbf{gesamte angeforderte Menge}}
\]
oder\\
\[
		\textbf{Lieferbereitschaftsgrad}=\frac{\textbf{sofort ausgelieferte Aufträge}}{\textbf{gesamte angeforderte Aufträge}}
\]
\\
\[
		\textbf{durchschn. Lagerbestand}=\frac{\textbf{Anfangsbestand + Endstand}}{\textbf{2}}
\]
\\
\[
		\textbf{Lagerumschlagshäufigkeit}=\frac{\textbf{Lagerabgang pro Jahr}}{\textbf{durchschn. Lagerbestand}}
\]
\\
\[
		\textbf{durchschn. Lagerdauer}=\frac{\textbf{durchschn. Lagerbestand}\cdot 360}{\textbf{Lagerabgang pro Jahr}}
\]
\\
\section{Optimale Bestellmenge}
\[
	x_{opt}=\sqrt{\frac{200\cdot M\cdot a}{p\cdot q}}
\]
\begin{footnotesize}
	$q:$ Zins-/Lagerkosten in [ \%]\\
	$M:$ Gesamtmenge pro Periode\\
	$a:$ Bestellfixkosten\\
	$p:$ Einstandspreis\\
\end{footnotesize}
\\
\section{Optimale Losgrösse}
\[
	x_{opt}=\sqrt{\frac{200\cdot M\cdot \left( H_{fix}+L_{fix} \right)}{h_{var}\cdot q}}
\]
\begin{footnotesize}
	$q:$ Zins-/Lagerkosten in [ \%]\\
	$M:$ Gesamtmenge pro Periode\\
	$H_{fix}:$ \\
	$L_{fix}:$ \\
	$h_{var}:$ 
\end{footnotesize}
\
\\
\chapter{Notizen}
\cleardoublepage
